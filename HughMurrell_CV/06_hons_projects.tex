% html: Beginning of file: `hons_projects.html'

\begin{description}
\item[Honours Project supervision] ( in approximate chronological order, 1988 -  ).
\item[ Reconstruction via the Hartley Transform ]
      David Carson derived the classic CAT reconstruction algorithm
          using the Hartley transform as the basic mathematical tool. He coded up his
          algorithms and obtained equivalent reconstruction results on test projection data.
\item[ Voxel based rendering ]
      Paul Melamed built a voxel based rendering system for producing
          3D pictures from CAT slices.
\item[ Ray tracing Laser Beams through Shock Waves ]
      Paul Baise built a computer model for generating the path traveled
          by rays of light through theoretical shock waves.
\item[ Ray Tracing for Mathematica ]
      Michael Haley built a ray tracing front end for Mathematica.
\item[ Speech Recognition and Sound Compression using Wavelets ]
      Richard De Oude carried out an investigation into the use of
          wavelet transforms as a tool for speech recognition and
          sound compression.
\item[ Bird Call Recognition ]
      Rowena Mannix used ideas from the speech recognition literature
          to build a system that attempted to recognize common bird calls.
\item[ Flight Plan Production     ]  
      Keith Crompton built a graphical system for light aircraft pilots to
          use when required to file flight plans for trips over Southern Africa.
\item[ Photo Faker ]
      Martin Fortmann developed an image processing systems that
          allowed the user to create fake photographs by inserting small images
          into larger ones and then smoothing out the joins.
\item[ Automated Mapping System ]
      Steven Clur built an image processing system that stitched together
          images taken from a helicopter while flying a predetermined path
          over an area of interest. The system was installed on the
          helicopter as a navigational aid to help the pilot stay on track.
\item[ Solid Modelling of Geological Phenomena]
      Sheila Van Der Willigen built a solid modeller (similar to noddy)
          for geological folding, shearing, slipping and weathering.
\item[ Web Based Bookings ]
      Cuan Brown developed a web-based database for travel agents to
          advertise tours and take bookings over the web.
\item[ 3D Julia Sets ]
      Mark Lewis built a system for displaying 3D Julia sets based on
          the convergence of quadratic quaternion generators.
\item[ Web Based Car Dealer Database ]
      Kamil Reddy built a web based car dealer database using new
          ideas from JAVA and JDBC.
\item[ Cane Simulation Web Service ]
      Yevern Govender studied an irrigation simulation package from the SA
          Sugar Cane Association Experiment Station and rewrote the simulator so that it could
          deliver irrigation programs to farmers over the web.
\item[ Interactive JPEG image compression ]
      Gavin Murrison constructed a program for
          the interactive compression of JPEG images.
\item[ Greyscale image enhancement using Pseudo-Colouring ]
      Keagan Moodley wrote a colour-map generator for providing
          pseudo-colour to greyscale images. One of his colour maps
          is now used by marine geologists to colour images of the sea-bed.
 \item[ LBW trainer for cricket umpires]
          Sean Patton used OpenGL to build an
          LBW trainer for cricket umpires.
          The model includes bowler batsman and wicket keeper.
          Sean built an algorithm for producing different deliveries that
          require LBW decisions.
          These are played at random and the user (the umpire)
          has his decisions recorded.
\item[ Algorithmic music environment ]
           Geoffrey Devantier made use of the computer music tool,
          {\em  Jmusic },
          to build an environment for constructing algorithmic music.
          The main feature of this system was a plugin facility which
          allows users to write simple algorithmic music generators.
\item[ Automatic music scorer ]
          Isacc Lundall tried to build a system that used wavelet
          transforms to convert sound samples to music scores. This was an
          ambitious project and only musical samples consisting of
          a sequence of pure notes was tackled.
\item[ Delaunay triangulation of the sphere]
          Jacqueline Maw used Mathematica and Java to construct an
          algorithm for generating Delaunay triangulations of points on
          the unit sphere. An algorithm for points on the plane already
          exists but triangulation (spherical) of points on the sphere
          is much harder (and in some cases impossible).
          Miss Maw tried to adapt the planar algorithm to the
          spherical case and deal with spherical nasties as they occur.
\item[ Delaunay triangulation of the sphere]
           Chris de Kadt repeated the work done by Miss Maw in 2003.
          Chris and I investigated a new triangulation algorithm
          for spherical data. The new algorithm overcame 
          some of the problems that occurred with Miss Maw's algorithm.
\item[ Folding proteins on lattice points]
           Kieran O'Neill investigated the protein folding problem.
          This problem is known as the {\em holy grail} of bioinformatics.
          Kieran investigated the application of genetic algorithms to
          the protein folding problem. 
\item[ Pitch Recognition Techniques using Fourier and Wavelet transforms]
          John McGuiness rewrote the {\em Tartini} tool
          to recognize single pitches in a given melody using both
          windowed Fourier transforms and Wavelet transforms. This
          work was much more successful than the note recognition system
          built by Isacc Lundall in 2002.
\item[ Automatic Motif Discovery]
    Stephen Pitchford, wrote and investigated {\em Mathematica} software for
    finding motif patterns in a set of nucleotide sequences.
\end{description}
\newpage
\label{f0}
% html: End of file: `hons_projects.html'
