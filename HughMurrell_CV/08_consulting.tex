% html: Beginning of file: `consulting.html'
          
 \begin{description}\item[] \begin{description}\item[] {\large \bf  Funding }
\begin{description}
 \item[ Bioinformatics     ]  
          I spent many hours during 2004 and 2005 preparing
          a funding application for a
          Bioinformatics node at UKZN. 
          I was the principal investigator
          for this fund and in cooperation
          with staff from our biochemistry department
          submitted the UKZN application
          late in 2005.
          We were awarded close to R1m early in 2006.
          During 2007 a further R0.5m continuation award was made.
          This award has been used to set up infrastructure and technical
          services to support the KwaZulu-Natal Bioinformatics Node and
          as a source of scholarships for bioinformatics students.
          
\end{description}
\end{description}
\end{description}
          
\begin{description}\item[] \begin{description}\item[] {\large \bf  Opensource }
\begin{description}
\item[ CRAN package ]
	During my 2012 sabbatical I wrote an R data mining package for discovering
	non-linear associations between variables in a dataset. Read 
	\url{https://journals.plos.org/plosone/article?id=10.1371/journal.pone.0151551}       
	for further details.
\item[ Covid-19 tracking ]
	During the first half of 2020 under lockdown, I constructed a Julia script
	to compute Rt estimates at scale from global data sets. This script updates
	nightly and allows users to compare Covid outbreaks from region to region.
	Results can be viewed here; \url{https://reproduction.live/}
\end{description}
\end{description}
\end{description}

\begin{description}\item[] \begin{description}\item[] {\large \bf  Consulting }
\begin{description}
 \item[ Leather Dyeing Recipes     ]  
          I connected a PC to an electronic scale using a BurrBrown
          board so as to monitor recipes for a local leather dyeing shop
          and update workshop inventories. 
 \item[ Noise Monitoring    ]  
          I wrote a vibration monitoring package for Toyota that listens to
          accelerometers placed at various parts of a car
          as it is taken through a rev sequence. 
          FFT's were calculated and plotted so as to
          enable engineers to track major frequencies versus rev count. 
 \item[ Unit Trust Performance     ]  
          I constructed a linux based database 
          containing equity performance
          predictions from a local stock-exchange research company. 
          I then wrote a front-end that queries the database
          and produces performance predictions for local unit trust portfolios.
 \item[ Product Counting    ]  
          I constructed product-counting software for
          a plastic injection molding company. This software produces daily,
          monthly and yearly graphs of production figures for each machine
          operated by the company.
 \item[ Access Control     ]  
          I act as consultant to Mr. Rob Davey who supplies 
          access control equipment and web-based monitoring
          devices. 
          
\end{description}
\end{description}
\end{description}

\newpage
\begin{description}\item[] \begin{description}\item[] {\large \bf  R Consulting }
\begin{description}
 \item[ Genetic Drift App   ]  During 2015 I constructed a ShinyApp
 that allows the user to upload a {\it wildtype} virus sequence  to a gene pool and then set {\it fitness} parameters before
 initiating an in-silico random genetic drift operation on the gene pool. The app employs third party software
 to perform RNA folding in parallel in order to implement one of the fitness parameters that the client was interested in.
 The app maintains a phylogenetic tree for all the sequences surviving in the gene pool and the app allows
 the user to download genetic variants of the wildtype from the gene pool for later in-vivo construction and testing.
 I wrote the app under instruction from Prof. Darren Martin of UCT's Faculty Of Health Sciences. 
 He plans to make the code open source eventually.

 \item[ Retirement Planning  App   ]  During 2015 and 2016 I constructed another ShinyApp
 that allows Financial Advisors to load their client's portfolio data and then use the app to simulate
 performance of the financial instruments into retirement and beyond. I wrote this application for Mr. Peter Strydom of Enhance IFA
 who intends to market the app to South African financial advisors. The app will be hosted on the shinyapps.io server
 and only paid up advisors will have access to it.
 
\end{description}
\end{description}
\end{description}

\label{f0}
% html: End of file: `consulting.html'
